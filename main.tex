\documentclass[12pt,letterpaper]{article}
\usepackage{fullpage}
\usepackage[top=2cm, bottom=4.5cm, left=2.5cm, right=2.5cm]{geometry}
\usepackage{amsmath,amsthm,amsfonts,amssymb,amscd}
\usepackage{lastpage}
\usepackage{enumerate}
\usepackage{fancyhdr}
\usepackage{mathrsfs}
\usepackage{xcolor}
\usepackage{graphicx}
\usepackage{listings}
\usepackage{hyperref}

\hypersetup{%
  colorlinks=true,
  linkcolor=blue,
  linkbordercolor={0 0 1}
}
 
\renewcommand\lstlistingname{Algorithm}
\renewcommand\lstlistlistingname{Algorithms}
\def\lstlistingautorefname{Alg.}

\lstdefinestyle{Python}{
    language        = Python,
    frame           = lines, 
    basicstyle      = \footnotesize,
    keywordstyle    = \color{blue},
    stringstyle     = \color{green},
    commentstyle    = \color{red}\ttfamily
}

\setlength{\parindent}{0.0in}
\setlength{\parskip}{0.05in}

% Edit these as appropriate
\newcommand\course{SIE 564}
\newcommand\hwnumber{2}                  % <-- homework number
\newcommand\NetIDa{Martin M Lopez}           % <-- NetID of person #1
%\newcommand\NetIDb{netid12038}           % <-- NetID of person #2 (Comment this line out for problem sets)

\pagestyle{fancyplain}
\headheight 35pt
\lhead{\NetIDa}
%\lhead{\NetIDa\\\NetIDb}                 % <-- Comment this line out for problem sets (make sure you are person #1)
\chead{\textbf{\Large Homework \hwnumber}}
\rhead{\course \\ January 29, 2019}
\lfoot{}
\cfoot{}
\rfoot{\small\thepage}
\headsep 1.5em

\begin{document}

\section*{Part A. Cost Estimate with COCOMO II}
Beaver, Inc. needs to develop a large new supply chain management system within 24 months and an available budget of \$8,200,000.  The full set of capabilities adds up to 200 KSLOC of software.  Beaver, Inc. is a CMMI Level 3 company, is generally familiar with the product line, will follow general conformity in design flexibility, generally does architecture and risk resolution, and has a largely cooperative team (i.e., all of its scale factors are rated High).  Its average labor cost is \$10,000, per person-month.  For simplicity, consider all of its baseline cost driver ratings to be Nominal.
\subsection*{Solution}
Given the problem statement we choose the Early-Design Effort Multipliers since Beaver Inc. is developing a large new supply chain management system. In determining the Person months required to develop a 200 KSLOC product of software we utilize the following Effort Equation:
    \begin{align*}
        &PM_{NS} = A * Size^E * \Pi_{i=1}^{n}EM_i\\
        &\text{where}\\
        &A = 2.94\\
        &Size = 200\\
        &E = B + 0.01 * \Sigma_{j=1}^{5}SF_j
    \end{align*}
We use $A = 2.94$ based on the COCOMO II.2000 Multiplicative calibrated effort coefficient.To determine $E$ which represents the scaling exponent for Effort we utilize the following equation:
    \begin{align*}
        &E = B + 0.01 * \Sigma_{j=1}^{5}SF_j\\
        &\text{Where}\\
        &B = 0.91\\
        &SF = \text{Scale Factors}
    \end{align*}
In the equation B represents the scaling base-exponent for effort that is calibrated. The five scale factors that we utilize are:
    \begin{enumerate}
        \item PREC = Prescedentness
        \item FLEX = Development Flexibility
        \item RESL = Architecture/Risk Resolution
        \item TEAM = Team Cohesion
        \item PMAT = Process Maturity
    \end{enumerate}
Due to the fact the Beaver Inc. is classified as company with a Capability Maturity Modeled (CMM) at a level 3 we match these scale factors to the following weights:
    \begin{enumerate}
        \item PREC = 2.48
        \item FLEX = 2.03
        \item RESL = 2.83
        \item TEAM = 2.19
        \item PMAT = 3.12
    \end{enumerate}
Given the values for each scale factor and utilizing $B = 0.91$ we calculate E.
    \begin{align*}
        &E = 0.91 + 0.01 * (2.48+2.03+2.83+2.19+3.12)\\
        &E = 1.0365
    \end{align*}
The next aspect of the Effort Equation is to determine the effort multipliers. Since this project is new and best fits the Early Design Effort Multipliers Criteria we select and match those values to those Effort Multipliers (EM).
We use this effort equation to later determine the Schedule utilizing the COCOMO II.2000 Schedule Equation:
    \begin{align*}
        &TDEV = C * (PM_{NS})^F * \frac{SCED\%}{100}\\
        &C = \text{Schedule coefficient that is calibrated} = 3.67\\
        &F = \text{Scaling exponent for schedule}
    \end{align*}
We calculate $F$:
    \begin{align*}
        &F = D + 0.2*[E-B]\\
        &D = \text{Scaleing Base exponent for schedule that's calibrated}=0.28\\
        &E = 1.0365\\
        &B = \text{Scaling base-exponent for the effort equation} = 0.91\\
        &F= 0.33+(0.2*(1.0365 - 1.01)) = 0.3355
    \end{align*}
We now have the equations to calculate the effort and the schedule to determine cost.
    \begin{align*}
        &PM_{NS} = (2.94)*(200)^{1.0365} * \Pi_{i=1}^{n}EM_i\\
        &TDEV = (3.67)*(PM_{NS})^{0.3355}* \frac{SCED\%}{100}
    \end{align*}
Compute the project’s cost and schedule for the following cases:
1.	Everything as described above, and RELY rated as Nominal\\
With the RELY rated as Nominal we have the scale factor of RECPX = nominal and the rest of the scale factors are nominal. We get the following:
    \begin{equation*}
        \Pi_{i=1}^{5} = 1.0
    \end{equation*}
We then obtain a Effort Number by using the effort equation:
    \begin{align*}
        &PM_{NS} = (2.94)*(200)^{1.0365} * \Pi_{i=1}^{n}EM_i\\
        &\Pi_{i=1}^{5} = 1.0\\
        &PM_{NS} = (2.94)*(200)^{1.0365} * (1.0) \\
        &PM_{NS} = 713.45 \approx 713 \\
        &Cost = 7,130,000\\
        &TDEV = (3.67)(713)^{0.3355}* \frac{100}{100} =  33.21 \approx 33  \text{ months}
    \end{align*}
We see that we meet the cost requirement as the cost is less than \$8,200,000 but do not meet the schedule requirement as more than 24 months.\\
2.	RELY rated as High\\
    \begin{align*}
        &PM_{NS} = (2.94)*(200)^{1.0365} * \Pi_{i=1}^{n}EM_i\\
        &\Pi_{i=1}^{5} = 1.10\\
        &PM_{NS} = (2.94)*(200)^{1.0365} * (1.10) \\
        &PM_{NS} = 784.79\approx 785\\
        &Cost = 7,850,000\\
        &TDEV = (3.67)*(785)^{0.3355}* \frac{100}{100} = 34.40 \approx 34 \text{ months}
    \end{align*}
In this scenario we still meet the cost requirement of \$8,200,000 but exceed the schedule requirement of 24 months.\\
3.	RELY rated as High, SCED rated as Low
    \begin{align*}
        &PM_{NS} = (2.94)*(200)^{1.0365} * \Pi_{i=1}^{n}EM_i\\
        &\Pi_{i=1}^{5} = 1.254\\
        &PM_{NS} = (2.94)*(200)^{1.0365} * (1.254) \\
        &PM_{NS} =  894.66\approx 895  \\
        &Cost = 8,950,000\\
        &TDEV = (3.67)*(895)^{0.3355}* \frac{85}{100} = 30.46 \approx 30 \text{ months}
    \end{align*}
In this scenario we do not meet neither the cost or schedule requirements.\\
Suppose the project prioritized its capabilities and decided to develop the top-priority requirements.  What would the 200 KSLOC need to be reduced to in order to satisfy all three “Cost, Schedule, Quality” criteria?\\

In reducing the amount of KSLOC to meet the cost schedule and criteria for all three scenarios must be updated to 80KSLOCs. The results are the following:\\
1. \\
    \begin{align*}
        &PM_{NS} = (2.94)*(200)^{1.0365} * \Pi_{i=1}^{n}EM_i\\
        &\Pi_{i=1}^{5} = 1.0\\
        &PM_{NS} = (2.94)*(80)^{1.0365} * (1.0) \\
        &PM_{NS} = 275.99 \approx 276  \\
        &Cost = 2,760,000\\
        &TDEV = (3.67)(276)^{0.3355}* \frac{100}{100} =  24.16 \approx 24 \text{ months}
    \end{align*}
2.\\
    \begin{align*}
        &PM_{NS} = (2.94)*(200)^{1.0365} * \Pi_{i=1}^{n}EM_i\\
        &\Pi_{i=1}^{5} = 1.0\\
        &PM_{NS} = (2.94)*(80)^{1.0365} * (1.0) \\
        &PM_{NS} = 303.59 \approx 303 \\
        &Cost = 3,030,000\\
        &TDEV = (3.67)(303)^{0.3355}* \frac{100}{100} =  24.9 \approx 25 \text{ months}
    \end{align*} 
3.\\
    \begin{align*}
        &PM_{NS} = (2.94)*(200)^{1.0365} * \Pi_{i=1}^{n}EM_i\\
        &\Pi_{i=1}^{5} = 1.0\\
        &PM_{NS} = (2.94)*(80)^{1.0365} * (1.0) \\
        &PM_{NS} = 346.09 \approx 346 \\
        &Cost = 3,460,000\\
        &TDEV = (3.67)(346)^{0.3355}* \frac{85}{100} =  22.15 \approx 22 \text{ months}
    \end{align*} 
\section*{Part B. Sanity check}
Chose an alternative estimation method (analogy, bottom up, or expert opinion, etc.) and discuss:\\
\quad \quad 1.	The process of how you would develop a cost estimate for the supply chain management system using this alternative method.\\\\
\quad The process to develop a cost estimate for the supply chain management system would be the bottoms up approach. Utilizing this methodology may involve many hours and planning but it will determine the the level of effort it would take to determine the development of the new software package. Experts would come and clearly detail out the Workbreak Down Structure (WBS) to determine how long it will take to complete the architecture, implementation, and testing of the software. The methodology will be determined whether the team would develop the software utilizing agile methodologies or waterfall. Based on these decisions the WBS will be structured, capabilities prioritized and work would be scoped.\\\\
Although this methodology is highly costly will help the program the team may develop a risk assessment and try to determine at what development points avoidance and mitigation must be identified. \\\\
\quad \quad 2.	How the cost estimate for the supply chain management system might differ by using this alternative method (Would it be higher or lower?  Have more or less detail?).\\\\
\quad The cost of utilizing this method will increase at the beginning of the project. Defining and defining the Workbreak down structure (WBS). Utilizing the waterfall method. The scope of the project will defined with the help of subject matter experts and team members. \\\\ 
\quad \quad 3.	Whether you expect this alternative method to yield a more accurate or less accurate estimate compared to the estimate obtained by using COCOMO II. \\\\
This method will yield a more accurate estimate as the workbreak down structure will be developed and all work estimated. The risk falls within capturing all work. This method will permit the team determine the various features and capabilities within the 200KSLOCs.
\section*{Part C. DMAIC}
    \begin{enumerate}
        \item Describe something you want to improve and have control over.\\\\
    I wish to reduce my body fat and Body Mass Index (BMI) within the next six months. This will allow me to live a healthier lifestyle. This will encompass increasing the amount of time exercising and measuring the caloric intake everyday. 
        \item Investigate features of the item(s) to be improved, decomposing the problem, and setting three targets for improvement: worse case, realistic case, and best case scenarios.\\\\
    The options to be improved include increasing the time exercising within a week. The amount of time and calories burned during a workout will be tracked. The utilizing a scale that measures body fat as well as a scale will measure weight, body fat, and BMI. The worse case scenario is maintaining body fat and BMI to the initial level. The realistic scenario is reducing my body fat percentage by 2-3\%. The best case scenario is reduce my body fat percentage by 5\%.\\ \\ 
        \item Identify source of variation and determine root causes.\\
    A major source of variation is the amount of caloric intake throughout a week. Adjusting diet and exercise routines is difficult to adjust and establish a consistent baseline. Developing a consistent workout routine will entail High Interval Intensity Training alongside cross training at the gym.\\ 
        \item Define improvement steps or interventions that will be implemented in order to reach desired targets in step 2.\\\\
    In order to achieve this goal, increasing the frequency of exercising within the week increases by 25\% within the second month will hopefully improve performance. The increase of 25\% of physical activity shall be completed after the first month after baselining the data.\\ 
        \item Identify mechanisms for long term sustainment. What rewards will incentivize high performance in the future? Identify commitment device for each of the three target levels (worst, realistic, best).\\\\
    The first level of commitment is essentially trying to meet the worse case scenario within the first three months. The realistic and best case scenario will support the goal of reducing my body fat for summer vacation. By this time I hope that I would establish a lifestyle. 
    \end{enumerate}
\end{document}
